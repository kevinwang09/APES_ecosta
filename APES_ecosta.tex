\documentclass[12pt,aspectratio=169]{beamer}

\usepackage{color}
\usepackage{listings}
\usepackage{amsmath}
\usepackage{amsthm}
\usepackage{amsfonts}
\usepackage{amssymb}
\usepackage{graphicx}
\usepackage{xcolor}
\usepackage{url}
\usepackage{moresize}

\usepackage{setspace}

\usepackage{tikz}
\usepackage{tikz-cd}
\usepackage{tikz-qtree}
\tikzcdset{row sep/huge=5em}
\tikzcdset{column sep/huge=4cm}
\usepackage[framemethod=TikZ]{mdframed}




\definecolor{themeBlue}{HTML}{3439B0}
\definecolor{myBlue}{HTML}{3439B0}

\usepackage{natbib}
\bibliographystyle{abbrvnat}
\setcitestyle{authoryear,open={(},close={)}}

\tikzset{
	invisible/.style={opacity=0},
	visible on/.style={alt={#1{}{invisible}}},
	alt/.code args={<#1>#2#3}{%
		\alt<#1>{\pgfkeysalso{#2}}{\pgfkeysalso{#3}}%
	}
}




\mdfdefinestyle{MyFrame}{%
    linecolor=black,
    outerlinewidth=1pt,
    roundcorner=5pt	,
    innertopmargin=\baselineskip,
    innerbottommargin=\baselineskip,
    innerrightmargin=20pt,
    innerleftmargin=20pt,
    backgroundcolor=white}



\setbeamercolor{block title}{bg=gray!60,fg=black}
\setbeamercolor{block body}{bg=blue!20,fg=black}



\usetheme{default}
%\usetheme{metropolis}
\usecolortheme{lily}
%\usecolortheme{whale}
\usefonttheme{professionalfonts}
\setbeamercovered{transparent} 
\setbeamertemplate{navigation symbols}{} 
\setbeamertemplate{itemize items}[circle]
\setbeamertemplate{section in toc}[square]
\setbeamertemplate{enumerate items}[default]




\newcommand{\bx}{\boldsymbol{x}}
\newcommand{\bX}{\boldsymbol{X}}
\newcommand{\bY}{\boldsymbol{Y}}
\newcommand{\by}{\boldsymbol{y}}
\newcommand{\bW}{\boldsymbol{W}}
\newcommand{\bZ}{\boldsymbol{Z}}
\newcommand{\bz}{\boldsymbol{z}}
\newcommand{\E}{\mathbb{E}}
\newcommand{\bbeta}{\boldsymbol{\beta}}
\newcommand{\bbetaHat}{\widehat{\boldsymbol{\beta}}}
\newcommand{\bbetaTilde}{\widetilde{\boldsymbol{\beta}}}
\newcommand{\bPi}{\boldsymbol{\pi}}
\newcommand{\bPiHat}{\widehat{\boldsymbol{\pi}}}
\newcommand{\bPiTilde}{\widetilde{\boldsymbol{\pi}}}
\newcommand{\hatPi}{\widehat{\pi}}
\newcommand{\bzHat}{\widehat{\bz}}
\newcommand{\alphaFull}{\alpha_{f}}
\newcommand{\piizero}{\pi_i^{(0)}}
\newcommand{\piit}{\pi_i^{(t)}}
\newcommand{\logit}{\operatorname{logit}}
%\usepackage[style=authoryear]{biblatex}


\title{APES: Approximated Exhaustive Search for GLM}

\author{Kevin Y.X. Wang \\ 
Twitter: @KevinWang009}



\begin{document}

%\begin{frame}
%\maketitle
%\end{frame}


\begin{frame}
\begin{center}
	{\Large \color{myBlue} APES: Approximated Exhaustive Search for GLM}
	
	\vspace{1cm}
	
	{\normalsize  Kevin Y.X. Wang}
	
	\vspace{0.5cm}
	
	{\footnotesize School of Mathematics and Statistics \\ The University of Sydney}

	\vspace{1cm}

	{\normalsize EcoSta - Taichung \\ 26 June 2019}

	\vspace{1cm}
\begin{columns}
	\begin{column}{0.5\linewidth}
		\flushleft{\includegraphics[scale=0.12]{usyd.png}}
	\end{column}
	\begin{column}{0.5\linewidth}
		\flushright {\scriptsize \includegraphics[scale=0.05]{Twitter.png}$@$KevinWang009}
	\end{column}
\end{columns}
\end{center}
\end{frame}


\begin{frame}{Acknowledgement}
\begin{itemize}
	\item This is joint work with Prof Samuel M\"{u}ller, Dr Garth Tarr and Prof Jean Yang.
	
	\item \texttt{mplot} \citep{Tarr2018} is a package to assess model stability and variable selection for linear models and generalised linear models.
\end{itemize}
\begin{figure}
	\centering
	\includegraphics[width=0.55\linewidth]{../../../figures/mplot_ad}
%	\caption{\texttt{mplot} is a package to assess model stability and variable selection for linear models and generalised linear models.}
	\label{fig:mplotad}
\end{figure}
\end{frame}


\begin{frame}
\Huge{\color{themeBlue} Background}
\end{frame}


\begin{frame}{Data and models}
\setstretch{1.5}
	\begin{itemize}
		\item $ \by = (y_1, y_2, \dots, y_n)^\top $, with independent $ y_i  \in \lbrace0, 1\rbrace$, $ i = 1, \dots, n $. 
		\item Design matrix $ \bX = (\bx_1, \dots, \bx_p) \in \mathbb{R}^{n \times p}$. 
		\item Index the columns of $ \bX $ by $ \lbrace 1, \dots, p \rbrace$. 
		\item 
		Let $ \alpha $ denote any subset of $ p_\alpha $ distinct elements from $ \lbrace 1, \dots, p \rbrace $. Use $ \mathcal{A} $ to denote the collection of all $ \alpha $, so $ |\mathcal{A}| = 2^p $.
		
		\item $ \bX_\alpha $ denote the $ n \times p_\alpha $ matrix with columns given by the columns of $ \bX $ whose indices appear in $ \alpha $.
	\end{itemize}
\end{frame}

\begin{frame}{Logistic regression}
\setstretch{1.5}
 \begin{itemize}
 	\item We model the conditional response variable $ Y_i | \bX$ as $ \operatorname{Bernoulli}(\pi_i) $, where $ \pi_i = \mathbb{P}(Y_i  = 1| \bX)$. 
 	\item We will use the \textbf{logistic function} as our link function, so $ \bx_i^\top \bbeta = \operatorname{logit}(\pi_i) = \ln \left( \pi_i/(1-\pi_i) \right)$.
 	\item Model fitting usually involves estimating $ \bbeta $ (or equivalently, $ \bPi $). 
 \end{itemize}
\end{frame}

\begin{frame}{Iterative Reweighted Least Square (IRLS)}
	\begin{enumerate}
	\item Denote weights $ w_i = \pi_i (1-\pi_i)$ and other estimates at the $ t $-th iteration with a superscript $ (t) $.
	\item Construct 
	\begin{equation*}
	z_i^{(t)} = \underbrace{\logit\left( \piit \right)}_{\bx_i^\top \bbeta^{(t)}}+ \frac{y_i-\piit}{\piit(1-\piit)}.
	\end{equation*}
	\item Update via
	\begin{equation*}
	\bbetaHat^{(t+1)} \leftarrow (\bX^\top \bW^{(t)} \bX)^{-1} \bX^\top \bW^{(t)} \bz^{(t)},
	\end{equation*}
	with $ \bW^{(t)} = \operatorname{diag}\left(w_i^{(t)}\right)$.
	
	\item At convergence, $ \bbetaHat = (\bX^\top \bW \bX)^{-1}\bX^\top \bW \bz$, which equals to the MLE of logistic regression model.
\end{enumerate}
\end{frame}







\begin{frame}{Challenges in exhaustive GLM variable selection}

For large $ p $, exhaustive variable selection in GLM is difficult:
	\begin{enumerate}
%		\item No close form for MLE, so numerical optimisations are needed.

\item The computational cost of IRLS is $ \mathcal{O}(np^2) $ per iteration.

\item We need to explore all $ 2^p $ models.
\end{enumerate}
\begin{figure}
	\centering
	\includegraphics[width=0.7\linewidth]{../../../figures/sloth}
	\label{fig:sloth}
\end{figure}



\end{frame}


\begin{frame}[fragile] %<-----------------------
	\frametitle{Aim of APES}
	\begin{mdframed}[style=MyFrame]
	Can we perform linear exhaustive variable selection (which benefits from fast algorithms) and use the results to approximate exhaustive GLM variable selection?
	\end{mdframed}
\vspace{1.5cm}
\begin{equation*}
\begin{tikzcd}[column sep=huge]
\text{GLM exhaustive}
& \text{LM exhaustive}  \arrow[l, "\text{Can this be done?}"', red]
\end{tikzcd}
\end{equation*}
\end{frame}

\section{APES: APproximated Exhaustive Search}


\begin{frame}
\Huge{\color{themeBlue} 1. Turning GLM to LM}
\end{frame}

\begin{frame}[fragile] %<-----------------------
	\frametitle{Exhaustively computing modified MLE}
	\setstretch{1.5}
	\begin{itemize}
		\item \citep{Hosmer1989} described an approximation to exhaustive variable selection for logistic regression without the need for numerical optimisation. 
		
		\item Their method starts with the estimated probability {\color{blue} $ \bPiHat(\alphaFull) $}, from the \textbf{full} logistic model.
		
		\item Then, for each model $ {\color{red} \alpha } \in \mathcal{A} $, we calculate:
		\begin{equation*} \label{eqn:biasedBetaHat}
		\hspace*{-0.7cm}
		\bbetaHat({\color{red}\alpha}; {\color{blue}\bPiHat(\alphaFull)}) = (\bX_{{\color{red}\alpha}}^\top \bW({\color{blue}\bPiHat(\alphaFull)}) \bX_{{\color{red}\alpha}})^{-1}\bX_{{\color{red}\alpha}}^\top \bW({\color{blue}\bPiHat(\alphaFull)}) \bz({\color{blue}\bPiHat(\alphaFull)}).
		\end{equation*}
		
		\item This is \textbf{NOT} the MLE for $ \alpha $, which should be $\bbetaHat({\color{red}\alpha}; \bPiHat({\color{red}\alpha})) $. 
		
%		\item It was proven that $\bbetaHat(\alpha_2; \bPiHat(\alpha_2)) = \bbetaHat(\alpha_2; \bPiHat(\alphaFull)) + \text{bias term}$.

	\end{itemize}
\end{frame}




\begin{frame}[fragile] %<-----------------------
	\frametitle{Variable selection using the modified estimator}
		\setstretch{1.5}
	\begin{itemize}
		\item Given $ \bbetaHat({\color{red}\alpha}; {\color{blue}\bPiHat(\alphaFull)}) $, we could \textbf{approximate} RSS or BIC for all $ {\color{red}\alpha} \in \mathcal{A} $. 
		
		\item Upon selection of a small set of desired models, we can recompute the MLE and calculate other model fit statistics. 
		
%\vspace{0.5cm}
%		\item These approximations provide a framework to approximate exhaustive variable search in the GLM space via an exhaustive model fitting in the LM space. 

\vspace{0.5cm}
\begin{equation*}
\begin{tikzcd}[column sep=huge]
\text{GLM exhaustive}
& \arrow[l, red, "\text{Hosmer approx.}", red] \text{LM exhaustive}
\end{tikzcd}
\end{equation*}

\end{itemize}

\end{frame}



\begin{frame}
\Huge{\color{themeBlue} 2. Reducing computational time}
\end{frame}


\begin{frame}{Best subsets search}
	\begin{itemize}
		
		\item Application of this approximation method is limited by the number of LMs we can explore.
		
		\item For $ p \approx 50 $, $ \mathcal{A}$ is approximately 1 quadrillion in size, which is too large to explore exhaustively. 
		
		\item \textbf{leaps} \citep{Furnival1974, Lumley2017} is a \underline{best subset} algorithm only exploring a subset of $ \mathcal{A} $ but guarantees to contain the global RSS-optimal model.
	\end{itemize}

\begin{figure}
	\centering
	\includegraphics[width=0.35\linewidth]{../../../figures/branch_and_bound}
	\label{fig:branchandbound}
\end{figure}
\end{frame}




\begin{frame}{Best subsets search}
		\setstretch{1.5}
\begin{itemize}
	
	\item Application is limited by the number of LMs we can explore.
	
	\item For $ p \approx 50 $, $ \mathcal{A}$ is approximately 1 quadrillion in size.
	
	\begin{mdframed}[style=MyFrame]
		A \underline{best subset} algorithm limits our search to a subset of $ \mathcal{A} $ but it guarantees to contain the global RSS-optimal model.
	\end{mdframed}
	
	\item Mixed Integer Optimisation (MIO) is the fastest best subset algorithm to date.
\end{itemize}
\end{frame}



\begin{frame}{Mixed Integer Optimisation}
\begin{itemize}
	\item \citep{Bertsimas2016} showed that it is feasible to perform best subset search for linear models with $ p $ in the hundreds.
	\item The most attractive component: guaranteed \textbf{sub-optimality} if algorithm is terminated before full convergence. Thus allowing a upper bound for real time limit.
%	\item Constraints using different loss functions and coefficient sizes are also possible. 
	\item Current implementation in \texttt{R} is \texttt{bestsubset}, \citep{Hastie2017}, which outputs the RSS-best linear model for each model size. 
\end{itemize}
\end{frame}



\begin{frame}[fragile] %<-----------------------
	\frametitle{APES: \underline{Ap}proximated \underline{E}xhaustive \underline{S}earch}
%	\begin{itemize}
%		\item There isn't such optimal algorithm for GLMs. 
%				
%		\item There is \textbf{no guarantee} that the best $ m $ linear models (as judged by BIC, say) will necessarily be the best $ m $ GLMs if an actual ES was performed in the GLM space. 
%		
%	\end{itemize}
	\begin{equation*}
	\begin{tikzcd}[column sep = huge, row sep = huge, ampersand replacement=\&]
	\text{GLM best subset} \arrow[visible on=<1>, d, "\text{no algorithm}"', blue]
	\& \text{LM best subset}   \arrow[visible on=<2>, d, "\text{MIO}", red] \\
	\text{GLM exhaustive} \arrow[visible on=<2>, u, red, "\text{infer}"]
	\& \arrow[visible on=<2>, l, red, "\text{Hosmer approx.}", red] \text{LM exhaustive}
	\end{tikzcd}
	\end{equation*}

\end{frame}








\section{Simulation}



\begin{frame}
\Huge{\color{themeBlue} Simulation}
\end{frame}


\begin{frame}{Simulation set-up}
	\begin{itemize}
		\item $ n  = 500,  p = 100$, number of non-zero coefficient is $ k = 6 $.
		
		\item Intercept term is set to 0, then we tried 3 different choices of $ \bbeta $: 
		\begin{enumerate}
			\item Equally spaced indices: $ (\frac{1}{2}, 0, \dots, 0,\frac{1}{2}, 0, \dots,0,\frac{1}{2}, 0, \dots, 0,\frac{1}{2}, 0, \dots, 0,\frac{1}{2}, 0, \dots, \frac{1}{2}) $.
			\item Block of indices, same magnitude/sign: $ (\frac{1}{2}, \frac{1}{2}, \frac{1}{2}, \frac{1}{2}, \frac{1}{2}, \frac{1}{2}, 0, \dots, 0) $.
			\item Block of indices, different magnitude/sign: $ (\frac{1}{3}, -1, 1, \frac{2}{3}, -\frac{2}{3}, -\frac{1}{3}, 0, \dots, 0) $.
		\end{enumerate}
		
		\item Generating $ \bX \sim \mathcal{N}(\boldsymbol{0}, \boldsymbol{\Sigma})$, with $ \Sigma_{ij} = \rho^{|i-j|} $, $ \rho = 0.5 $ or $ -0.5 $. Then standardise. 
		
		
		
		
		\item We repeated the simulation 100 times and compared APES against de-biased Lasso using various evaluation metrics.
	\end{itemize}
\end{frame}

%\begin{frame}{Simulation summary}
%\begin{itemize}
%	\item Fix the number of non-zero coefficients to be 6 (a sparse setting):
%	
%		\begin{itemize}
%			\item APES considered RSS-best models for size from 1 to 11.
%			\item Lasso considered models with 1000 $ \lambda $ values. 
%		\end{itemize}
%	
%	\item APES performed well when:
%		\begin{enumerate}
%			\item correlation is `mixed', with $ \rho = -0.5 $. 
%			\item non-zero coefficients are grouped together. 
%		\end{enumerate} 
%	
%	\item APES has the ability to select informative models in the event of hitting time limit. 
%\end{itemize}
%\end{frame}


%
%
%\begin{frame}{Evaluation 1: model path}
%For every model size, we have a model considered by both APES and lasso. 
%\begin{figure}
%	\centering
%	\includegraphics[width=1\linewidth]{../../../figures/ibsarScBicModelPlot1_2017_Nov_09}
%\end{figure}
%
%\end{frame}
%
%
%\begin{frame}{Evaluation 1: model path}
%We are able to identify the true model size, even in the case of hitting time limit. 
%\begin{figure}
%	\centering
%	\includegraphics[width=1\linewidth]{../../../figures/ibsarScBicModelPlot2_2017_Nov_09}
%\end{figure}
%\end{frame}
%




\begin{frame}{Evaluation 1: saturated models}

\begin{itemize}
	\item For each model size, APES and Lasso outputs one model. 
	\item In most cases, APES has less false exclusion of variables than Lasso. 
\end{itemize}

\begin{figure}
	\centering
	\includegraphics[width=0.8\linewidth]{../../../mod_sel_simulations/figures/ecostaCorrectModelPlot_2018_May_11}
\end{figure}


\end{frame}



\begin{frame}{Evaluation 2: variable selection}

\begin{itemize}
	\item BIC was used to select an optimal model. 
	\item In most cases, APES has higher selection probability of active variables than Lasso. 
\end{itemize}

\begin{figure}
	\centering
	\includegraphics[width=0.8\linewidth]{../../../mod_sel_simulations/figures/ecostaSelectedModelBetaPlot_2018_May_11}
\end{figure}
\end{frame}


\begin{frame}{How fast is APES compare to genuine exhaustive search?} 
Very. 
%\begin{figure}
%	\centering
%	
%	\visible<2>{
%	\includegraphics[width=0.4\linewidth]{../../../mod_sel_simulations/figures/leapsTimePlot_EcoSta_17_Apr_2019}
%	}
%\end{figure}

\begin{columns}
\begin{column}{0.5\linewidth}
\begin{figure}
	\centering
	\includegraphics[width=\linewidth]{../../../mod_sel_simulations/figures/goldStandardTimePlot_EcoSta_2019_Jun_24}
\end{figure}
\end{column}
\begin{column}{0.5\linewidth}
\begin{figure}
\centering
\visible<2>{
\includegraphics[width=\linewidth]{../../../mod_sel_simulations/figures/leapsTimePlot_EcoSta_17_Apr_2019}
}
\end{figure}
\end{column}
\end{columns}
\end{frame}

\begin{frame}{Some extensions}
The choice of $ {\color{blue}\bPiHat(\alphaFull)} $ can be relaxed in two different ways: 
\begin{itemize}
	\item The full model $ \alphaFull $ is not necessarily the best for variable selection. 
\begin{figure}
	\centering
	\includegraphics[width=0.8\linewidth]{../../../mod_sel_simulations/figures/varSelectionPlot_2018_Sep_11}
\end{figure}
	\item We could replace the MLE by other estimators, e.g. the Lasso or robust quasi-likelihood estimator.
\end{itemize}
\end{frame}




\begin{frame}{Final remarks}
\begin{enumerate}
	\item APES is a \textbf{fast approximation} method for \textbf{exhaustive} variable selection in GLM. 
	\item APES pushes model dimensions into the hundreds/thousands and serves as a standard of comparison like a true exhaustive search. 
	\item APES is provisionally accepted by Australia \& New Zealand Journal of Statistics as an invited paper (14 hours ago): 
	\begin{itemize}
		\item \href{https://github.com/kevinwang09/APES}{https://github.com/kevinwang09/APES}
		\item \href{https://github.com/garthtarr/mplot}{https://github.com/garthtarr/mplot}
%		\item \href{https://github.com/kevinwang09/APES}{https://github.com/kevinwang09/APES}
		\item Email: \href{mailto:kevin.wang@sydney.edu.au}{kevin.wang@sydney.edu.au}. Twitter: \href{https://twitter.com/KevinWang009}{@KevinWang009}
	\end{itemize}
   
    \item Statistical Society of Australia has generously sponsored my travel to Taichung. 
\end{enumerate}
\end{frame}



\begin{frame}{References}
\bibliographystyle{apalike}
\scriptsize
\bibliography{APES_ref.bib}
\end{frame}

\end{document}